\chapter{Problématique scientifique du stage}
\begin{large}
\textbf{Développement de méthodes pour la caractérisation hydromécanique de matériaux hétérogènes}\\
\end{large}
\break
La caractérisation des propriétés hydromécaniques des matériaux hétérogènes (matériaux constitués par définition d'au moins 2 éléments de natures différentes tels que la bentonite) constitue la problématique scientifique générale des travaux présentés dans ce rapport. Sa résolution est d'intérêt pour de nombreuses applications parmi lesquelles on peut citer l'étude des sols, le contrôle santé des ouvrages d'art ou des sites de stockage de déchets radioactifs.\\
Un des problèmes rencontrés dans les ouvrages d'art ou les sols est l'endommagement des structures. En effet, ces endommagements peuvent être par exemple des affaissements de terrain, l'apparition de fissures, le vieillissement,causés par une multitude de facteurs comme le climat, la sollicitation excessive. Ces problèmes pourraient être évités ou du moins être suivis et prévenus à temps.\\
Les problématiques dans le domaine du stockage des déchets radioactifs sont également d'ordre sécuritaire. Nous allons expliciter le principe du stockage des déchets radioactifs du fait de notre collaboration avec l'ANDRA.\\
\linebreak
\linebreak
\textbf{Stockage des déchets}\\
\linebreak
\linebreak
Ces sites de stockage, qui représentent un véritable enjeu écologique, sont un ensemble de galeries souterraines à plusieurs centaines de mètres de profondeur percées dans de l'argilite.
Les argiles gonflantes, parmi lesquelles on trouve la bentonite, sont utilisées dans les ouvrages de fermeture. Ces scellements empêchent un éventuel flux d'eau porteur de radionucléides, qui sont des atomes rayonneurs. Les scellements sont constitués d'un noyau de bentonite avec un confinement mécanique entre deux massifs de béton. Initialement, la bentonite est sous la forme d'un assemblage de matériaux granulaires très hétérogène qui par la suite va s'hydrater au contact de la roche environnante pour aboutir, au final, à un scellement de faible perméabilité à l'eau avec une pression de gonflement de quelques MPa. La bentonite va gonfler à mesure qu'elle sature en eau et devenir un matériau homogène.\\
Caractériser les propriétés de matériaux nécessite le développement de méthodes faisant appel à de l'instrumentation, de la modélisation et à la résolution d'un problème inverse. Les méthodes en question doivent être adaptées aux matériaux et structures à caractériser et sont donc très souvent corrélées au contexte applicatif et aux contraintes que celui-ci peut imposer. C'est le cas pour les travaux présentés dans ce rapport qui portent sur la conception d'une sonde capable de détecter les hétérogénéités de la phase initiale de la phase initiales puis de suivre le processus d'homogénéisation du matériau. Cette sonde pourrait également être utilisée dans une autre configuration pour le contrôle santé des ouvrages d'art.\\


