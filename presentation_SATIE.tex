\chapter{Environnement du stage}
Les travaux présentés dans ce rapport ont été effectués dans le cadre d'une collaboration entre plusieurs équipes de recherche appartenant au laboratoire Système et applications des technologies de l'information et de l'énergie (SATIE UMR CNRS 8029), au Commissariat à l'énergie atomique (CEA) de Saclay, à l'Agence nationale pour la gestion des déchets radioactifs (ANDRA) (IS0 9001,ISO 14001, OHSAS 18001) et au laboratoire de mécanique et de technologie (LMT-CACHAN UMR 8535).\\
L'objet de cette collaboration est une étude dans le domaine des capteurs et de l'instrumentation. Il s'agit plus précisément de concevoir une méthode de caractérisation de matériaux hétérogènes basée sur une technique de spectroscopie large bande.\\
Ces travaux s'inscrivent dans le cadre d’un projet scientifique intitulé RFhydroméca soutenu par le Labex LaSIPS qui est une structure qui développe une vision pluridisciplinaire et une approche « système » autour de 3 domaines : l'Electrical Engineering, le génie mécanique ainsi que le génie biologique.\\
Pour ce stage j'ai été accueilli dans les locaux du laboratoire SATIE, sur le site de l'ENS Cachan, au sein de l'équipe instrumentation et imagerie. J'y ai disposé des moyens de simulation et d'expérimentation propres à la conception RF qui était l'objectif principal du stage.\\
L'équipe d'encadrement du stage était constituée pour les personnes dépendantes du laboratoire SATIE d'Eric Vourch, maître de conférence HDR à l'ENS Cachan, de Frank Daout, maitre de conférence à l'IUT de ville d'Avray ainsi que de Dominique Placko ainsi Thierry Bore, qui sont respectivement professeur des université à l'ENS Cachan et ingénieur de recherche contractuel. Pour ce qui est du CEA, l'encadrement a été assuré par Claude Gatabin. Sylvie Lesoille, de l'ANDRA a également assuré l'encadrement, tout comme Mohand Chaouche, directeur de recherche CNRS du LMT.Par ailleurs, au quotidien j'ai côtoyé les chercheurs, les doctorants, les stagiaires et le personnel technique présents sur le site du laboratoire SATIE.
\section{Les équipes partenaires du projet LaSIPS RFhydromeca}
\subsection{CEA}
Le CEA\cite{ref2} est un organisme de recherche intervenant dans 4 grands domaines : les énergies basses carbones (nucléaires et renouvelables), les technologies pour l'information et les technologies pour la santé, les Très grandes infrastructures de recherche (TGIR), la défense et la sécurité globale. Le CEA est implanté sur 10 centres répartis dans toute la France dont 3 en Ile de France.\\
\subsection{ANDRA}
L'ANDRA\cite{ref3} a été créée en 1979 et est un établissement public industriel et commercial et placée sous la tutelle des ministères de l'énergie, de la recherche et de l'environnement. Il s'agit d'un organisme chargé de la gestion à long terme des déchets radioactifs produits en France
\subsection{SATIE}
Le laboratoire SATIE\cite{ref1}, est un laboratoire de recherche en sciences appliquées opérant principalement dans les domaines de l'Electrical Engineering. Fondé dans les années 70, il emploie aujourd'hui 192 personnes reparties sur plusieurs sites géographiques en Ile de France. \\
Le laboratoire est divisé en 2 pôles de recherche qui eux mêmes sont constitués de groupes de recherche:
\subsubsection{Composants et systèmes pour l'énergie électrique (CSEE)}
Ce pôle de recherche est axé sur l'étude de systèmes jusqu'aux composants et aux matériaux qui le constituent. Dans ce pôle, on peut discerner 3 groupes de recherche répartis sur 5 sites:\\
\begin{itemize}
\item Électronique de puissance et intégration (EPI)
\item Matériaux magnétique pour l'énergie (MME)
\item Système d'énergie pour les transports et l'environnement (SETE)
\item Technologies nouvelles (TN)
\end{itemize}
\subsubsection{Système d'information d'analyse multi-échelles(SIAME)}
Ce pôle de recherche se focalise sur l'instrumentation et le diagnostic de systèmes à différentes échelles. 3 groupes de recherche constituent ce pôle:\\
\begin{itemize}
\item Bio-microsystèmes et bio-capteurs (BIOMIS)
\item Méthodes et outils pour les signaux et systèmes (MOSS)
\item Instrumentation et imagerie (II)
\end{itemize}
Mon stage s'effectue au sein du pôle SIAME dans le groupe II.
\section{Présentation du groupe II}
Le groupe Instrumentation et Imagerie (II) a pour objectifs d'observer et de diagnostiquer les systèmes physiques et vivants. Afin de répondre à ces objectifs, le groupe développe des systèmes d'instrumentation pour l'observation des systèmes physiques et vivants (capteurs, multicapteurs et imageurs) mais aussi développe la modélisation des phénomènes physiques tout comme les algorithmes de traitement des signaux destinés à la caractérisation des systèmes et matériaux observés (détection, estimation de paramètres, diagnostic).
