\documentclass[a4paper, 12pt]{report}

% entête
\usepackage[intlimits]{amsmath}
\usepackage{amsfonts}
\usepackage{amssymb}
\usepackage{array}
\usepackage{makeidx}
\usepackage{xcolor}
\usepackage{pifont}
\definecolor{lightgray}{gray}{0.5}
\usepackage[utf8]{inputenc} 
\usepackage[T1]{fontenc}
\usepackage{lmodern}
\usepackage{here}
\usepackage{graphicx}
\usepackage[french]{babel}
\usepackage[left=2cm,right=2cm,top=2cm,bottom=2cm]{geometry}
\usepackage{tikz} 
% nom
\author{Jianpeng XU}
\usepackage{fourier}	
\usepackage{hyperref}
\usepackage{listings}
\usepackage{fancyhdr}
 \usepackage{enumitem}
% entête

\pagestyle{fancy}


%raccourcis
\newcommand{\diff}{\mathop{}\mathopen{}d}
\newcommand{\dphi}{\diff \phi}
\newcommand{\du}{\diff u}
\newcommand{\dv}{\diff v}
\newcommand{\alal}
{\linebreak
\linebreak
\linebreak
\linebreak}

\newcommand{\fig}[3]{
\begin{figure}[!h]
\centering
\includegraphics[scale={#1}]{{#2}}
\caption{{#3}}
\end{figure}
}

\lstset{language=C++,keywordstyle=\color{orange}}

\renewcommand{\headrule}{{\color{lightgray}\rule{\textwidth}{0.5pt}}}
\fancyhead[]{}
\fancyhead[R]{\textcolor{lightgray}{\leftmark}}

\makeatletter
\def\footrule{{
  \vskip-\footruleskip\vskip-\footrulewidth
  \color{\footrulecolor}
  \hrule\@width\headwidth\@height
  \footrulewidth\vskip\footruleskip
}}



%  bas des pages
\makeatother
\renewcommand{\footrulewidth}{0.5pt}
\newcommand{\footrulecolor}{lightgray}
\fancyfoot[C]{\textcolor{lightgray}{Rapport de stage de Jianpeng XU - Année universitaire 2016/2017}}
\fancyfoot[L]{}
\fancyfoot[R]{\textcolor{lightgray}{\thepage}}
\renewcommand{\chaptermark}[1]{ \markboth{#1}{} }
\renewcommand{\sectionmark}[1]{ \markright{#1}{} }

\begin{document}
\makeatletter
\let\ps@plain\ps@fancy
\makeatother
\begin{center}
\LARGE



% ------------------ Page de garde ----------------------




\begin{flushright}
\includegraphics[scale=0.85]{psud.jpg} 
\end{flushright}

\huge
\textbf{Rapport de Stage de fin d'étude } 
\linebreak
\normalsize
\alal
\huge

%titre
\textbf{Sondeur de canal MIMO à bipolarisation}
\\Effectué dans le laboratoire SATIE
\linebreak
\linebreak
\includegraphics[scale=0.8]{SATIE.jpg}
\includegraphics[scale=0.35]{logo_ensc_bleu.png} 
\alal
\vfill
\large
Encadré par Mme Isabelle VIN et M Jean-pierre BARBOT\\
Tuteur académique : M Hugues MOUNIER\\
Jianpeng XU\\
mars 2017 - Septembre 2017
\alal

\end{center}
\newpage
\setcounter{tocdepth}{3}
\tableofcontents
\newpage
\listoffigures
%\listoftables
\newpage


%------------- Introduction -------------
\fancyhead[]{}
\fancyhead[R]{\textcolor{lightgray}{Introdution}}
\begin{Huge}
\textbf{Résumé}:\\
\end{Huge}
\break


Face à l'expansion de l'utilisateur du systèmes de communications mobile, la rapidité et le débit est un facteur clé pour maintenir la compétitivité des fournisseur de d'accès internet 4G. Le signal transmis subit des perturbations entre la base et le mobile. D'où l'importance de connaitre le canal de propagation entre l'émetteur et le récepteur pour optimiser le signal à la réception. Dans cet optique je suis en charge tout d'abord de remettre en marche un sondeur de canal qui actuellement n'est pas en état de fonctionnement. Finalement concevoir une version MIMO de la base qui permet d'augmenter le débit en fonctionne de nombre d'antenne utilisé. Et aussi de mettre à jour la base tant au niveau matérielle tant au niveau logicielle.
\\
\alal
\alal


%-------------- Remerciment ------------------
\begin{Huge}
\textbf{Remerciements}:\\
\end{Huge}
\break

Je tiens tout d'abord à remercier infiniment madame Isabelle VIN et monsieur Jean-pierre BARBOT pour m'avoir offert la possibilité de faire ce stage, de m'avoir très bien accueilli et de m'avoir aidé et conseillé tout au long de ce stage.\\
Un très grand remerciement à l'ensemble des personnes qui m'ont accueillis au sein du laboratoire SATIE.



\newpage
\fancyhead[R]{\textcolor{lightgray}{\leftmark}}
\chapter{La remise en conformité du Sondeur de canal}

Le sondeur de canal est réalisé par mon tuteur M Jean-pierre BARBOT en 1995 dans le cadre du projet CNET RAMEAU. Ce sondeur de canal a pour fonction de caractériser le canal de propagation afin de fournir des études de cas,des statistiques et des modélisation du canal. Ceci est utilisé lors de la dimensionnement de la puissance de la station de base et des équipement mobile. A l'époque où les sondeurs de canal ont une dimension immense échelle d'un camion, ils ont crée un sondeur de dimension 1,5m*0.8m*0.8m. Ce qui leur permet de réaliser des essais à l'intérieur des bâtiment. Ceci a pour but de couvrir les réseau mobile à l'intérieur des bâtiments.Ensuite le sondeur a été modifié pour un autre projet après quelques années, les modifications apportés a causé des soucis. Une partie de la commande d'acquisition est endommagée.  \\

Dans la première partie nous nous intéressons tout d'abord à analyser les effets perturbateurs subis par le signal dans le canal de propagation et sa modélisation choisie. Ainsi que l'identification des matérielles constituées par le sondeur. Par la suite commence les essais sur le sondeur pour identifier si une autre partie est tombé en panne. Pour conclure l'objectif final de cette partie consiste à remplacer les composants hors-service, de mettre à jour les composants qui sont obsolètes et également d'optimiser l'espace utilisée pour réduire la taille du sondeur. Pour ce faire il faut tout d'abord optimiser l'espace vide entre les composants, puis il faut voir s'il est intéressant de remplacer les composants encombrants par celles qui ont la même fonctionnalité mais occupe moins d'espace. Dans de remettre le sondeur en état de marché.

\section{Caractérisation du canal de propagation}

Nous souhaitons établir une radiocommunication entre une base et mobile dans la bande de U.H.F.      \ (1-3 GHz). Il y a une fluctuation entre le signal reçus et émis, ceci est dû aux différents effets de perturbation introduits par le canal de propagation. Ces effets sont détaillés dans la suite de la section "Effets du canal de propagation".\\ 
\newpage
\subsection{Canal de propagation}

Le canal de propagation est le lien physique entre l'émetteur et le récepteur des signaux.  Les éléments situés dans le canal de propagation réduisent l'amplitude du signal, en cause ces éléments ont un indice de radio-opacité qui réduit la puissance du signal. La zone de propagation du signal dépend de l'angle d'émission de l'antenne émettrice. L'antenne réceptrice peut recevoir des différentes versions du signal qui a un délai ou déphasé par rapport au signal initialement émis. En effet lorsque les ondes entrent en contact avec des objets réfléchissants dans le canal,ils sont dérivés et changés de direction de propagation. Par conséquent les ondes ne sont pas initialement orientées dans la direction de l'antenne réceptrice vont finalement atteindre cette antenne. On appelle ce phénomène propagation multi-trajet. Les chemins de parcours des ondes sont illustrés sur la figure suivante.\\
\fig{1}{trajetm.png}{Phénomène du trajet multiple}

%bibliohttp://https://www.slideshare.net/ShreeKrupa1/multichannel-fading

\begin{itemize}[label=\ding{93}, font=\large \color{lightgray}]
\item La réflexion : lorsque l'onde électromagnétique du signal rencontre dans sa direction une surface lisse dont les dimensions sont grandes par rapport à la longueur d'onde du signal.

\item La diffusion : lorsque l'onde électromagnétique du signal entre en collision avec une surface dont la surface n'est pas parfaitement plane et lisse. Ce phénomène engendre la diffusion de l'onde dans plusieurs directions.

\item La diffraction : lorsque l'onde électromagnétique heurte une arête d'un corps volumineux dont les dimensions sont grandes (les bâtiments, les collines) par rapport à la longueur d'onde  du signal. Le signal contourne l'obstacle et continuer à se propager derrière celui-ci accompagnée d'une atténuation de la puissance.\\
\end{itemize}

C'est pour cette raison que les signaux reçus par les antennes peut être issue de ces deux chemins différent:\\
\begin{enumerate}
\item Le signal direct (Line of sight en anglais) : c'est le signal qui a parcouru le chemin de propagation le plus court. 
\item Le signal indirect (No Line of sight en anglais) : le signal qui a été réfléchit,diffracté ou diffusé par un élément dans le canal de propagation. Il existe une atténuation de la puissance et une latence entre le moment de la réception du signal direct et ce signal,en raison de chemin de propagation plus longue.\\
\end {enumerate} 

Dans la plupart des cas le signal indirect est considéré comme une interférence qui perturbe la reconstruction du signal émis. Dans le cas où le signal indirect a une déphasage d'un multiple d'une demi-période par rapport au signal émis,le signal reçus est fortement atténué par celui-ci. Dans le cas où le signal indirect a une déphasage d'un multiple d'une période, le signal est amplifié. Et dans le cas où un obstacle de grande est sur le chemin direct et atténue fortement ou masque complètement le signal. Le signal indirect devient très utile pour reconstruire le signal émis.


\subsection{Effets introduits par le canal de propagation}
\label{Effets}

Lors de la réception des signaux, on constate qu'il y a une recombinaison des signaux directs et indirects. Le signal reçus subit des phénomènes de perturbation qu'on peut catégoriser par la variabilité spatiale et la sélectivité fréquentielle.

\subsubsection{Variabilité spatiale}

La variabilité spatiale a pour effet d'atténuation de la puissance du signal reçus et d'introduire des erreurs. On catégorise cet effet en 3 échelles de fluctuation.\\
\begin{enumerate}
\item Les évanouissements rapides qui sont l'origine des paquets d'erreurs introduisent lors d'une communication numérique. Dans le cas de la communication analogique ils sont la cause de la modulation de fréquence aléatoire qui perturbe le système.

\item Une affaiblissement dite moyenne échelle. La cause d'influence est l'environnement du canal de propagation. Les bâtiments situés dans le canal produisent un effet de masque qui atténue le signal. La météo peut également avoir un impact. Lors que 

\item Une affaiblissement dite grande échelle est dû essentiellement aux déplacement du récepteur mobile, l'amplitude du signal change en fonction de la distance sépare de l'émetteur et le récepteur.

\end{enumerate}
\subsubsection{Sélectivité fréquentielle}

La sélectivité fréquentielle que manifeste le canal est dû également au trajet multiple du signal. Un très grand débit impose une grande bande passante et si cette bande passante "couvre" une partie du spectre comportant des creux (dus aux trajets multiples), il y a perte totale de l'information pour la fréquence correspondante. Lorsque le débit de transmission est élevé les signaux indirects reçus vont gêner la reconstitution du signal émis lors de la réception des signaux par l'antenne. En conséquence pour que le signal soit bien transmis il faut que la durée des symboles (c'est-à-dire le temps qui sépare 2 séquences de N données) adapte à la dispersion des retards. C'est l'écart-type de la fonction densité de puissance($P_h$), elle est exprimé par:\\
\[ P_h(0,\tau)=\lim_{t->\infty}(\frac{1}{T}\int_{\frac{T}{2}}^{-\frac{T}{2}}h(t_2,\tau)h^{*}(t_2,\tau)dt_2) \]

 Ce qui vas limiter le débit maximum. Sur la figure suivante on constate une partie du signal initial émis est recouvert par les signaux indirects et déphasés et ainsi la reconstruction du signal de base peut ne pas être réalisable.
%\fig{0.85}{écho.png}{Phénomène du trajet multiple}
\\
\subsection{Mesure du canal de propagation}

La fonction principale du sondeur est de caractériser le canal. Dans cette partie nous allons voir le modèle choisi du canal et comment isoler pour obtenir la caractérisation du canal.

\subsubsection{Modèle du canal de propagation}	
 
Nous souhaitons mettre les effets introduits dans le canal sous équation. La modélisation choisi est issue de la modélisation dite boite noire. \\
\begin{center}
\begin{tikzpicture}
\node at (0,1.5) {Antenne émettrice};
\draw (0,1)--(2,1);
\fill[black] (2,0) -- (2,2) -- (5,2) -- (5,0);
\node at (3.5,1)[white] {Boite noire};
\draw (5,1)--(7,1);
\node at (7,1.5) {Antenne réceptrice};
\end{tikzpicture}
\end{center}
L'étude des équations de Maxwell réagissant aux effets montre que ces effets induits dans les différents version du signal est linéaire. Dans le cadre de notre mesure, le temps d'échantillonnage est relativement faible,
on suppose que les caractéristiques du canal n'évolue pas pendant le temps de mesure.Donc nous modélisons le canal en utilisant le modèle de la boite noire par un filtre linéaire invariant, sous la forme suivante : \\
\[ s(t)= \int_{+\infty}^{-\infty} I(t,\tau)*e(t- \tau)\,\mathrm{d}\tau \]
avec
\begin{itemize}[label=\ding{93}, font=\large \color{lightgray}]
\item e(t): le signal émis
\item s(t): le signal reçu
\item $I(t,\tau):$ la réponse impulsionnelle du canal à l'instant t pour une impulsion de Dirac émise à l'instant$(t-\tau).$
\end{itemize}
Sous l'enveloppement complexe s'exprime de cette façon:
\[ y(t)= \int_{+\infty}^{-\infty} h(t,\tau)*x(t- \tau)\,\mathrm{d}\tau \]
avec
\begin{itemize}[label=\ding{93}, font=\large \color{lightgray}]
\item x(t): l'enveloppe complexe du signal émis
\item y(t): l'enveloppe complexe du signal reçu
\item $h(t-\tau) \text{l'enveloppe complexe de } \frac{1}{2}I(t-\tau)$ 
\end{itemize}
Nous considérons que les effets de propagation sont aléatoire donc la modélisation est construite sous l'hypothèse de WSSUS.

\subsubsection{Méthode choisie pour réaliser la mesure}

A partir de l'équation sous l'enveloppement complexe et la propriété du produit de convolution fait que la convolution des signaux peut s'exprimer de cette manière :

\[ R_{xy}(t)= \int_{+\infty}^{-\infty} h(t,\tau)*R_{xx}(t- \tau)+R_{xb}(t) \]
 avec 
\begin{itemize}[label=\ding{93}, font=\large \color{lightgray}]
\item $ R_xy(t) $: la corrélation du signal émis avec le signal reçu
\item $ R_xx(t) $: l'auto-corrélation du signal émis
\item b(t): 	\quad le bruit contenu dans la mesure
 %\[ R_xy(t) \text{La corrélation du signal émis avec le signal reçu}\]
 %\[ R_xx(t) \text{L'autocorrélation du signal émis}\]
 %\[ b(t) \text{b est le bruit contenu dans la mesure}\]
\end{itemize}

Le cas le plus simple pour les  où Rxx(t) est une impulsion de Dirac. La convolution de Rxx(t) avec h(t) donne tout simplement h(t) en raison de la propriété neutre du Dirac face à une convolution. On sait que l'auto-corrélation du bruit blanc est une impulsion Dirac. Mais le soucis est que lors des mesures il est impossible d'utiliser un signal qui a une durée infinie et non périodique. Donc il convient d'utiliser un signal fini périodique qu'on appelle un Pseudo-bruit qui donc sa fonction d'auto-corrélation est d'un triangle pointu. 
\fig{0.3}{autoc.png}{auto-corrélation du pseudo bruit}\\
Et sa densité spectrale de puissance est la suivante, en exploitant la zone autour de 0, on peut le considéré comme la DSP d'un bruit blanc.
\fig{0.3}{dsp.png}{Densité spectrale de puissance du pseudo bruit}\\
La méthode qui permet de générer une séquence de Pseudo-bruit est donné dans la figure suivante.
\label{gene code pseudo}
\fig{0.8}{pseudo.pdf}{Génération de la suite de donnée pseudo aléatoire}\\
Une partie des suites de bouclage de la contre-rétroaction nécessitant qu'une seul opération ou exclusif avec la sortie afin d'obtenir les séquence de donnés est donné dans ce tableau:
\fig{0.5}{shifttable.jpg}{Tableau de la contre-rétroaction}\\
%label=\textbullet
\begin{itemize}[label=\ding{93}, font=\large \color{lightgray}]
\item n : le nombre de registre utilisé
\item m : le bouclage de la contre-rétraction sur ce registre
\item n-m : une autre possibilité de bouclage de registre
 %\[ R_xy(t) \text{La corrélation du signal émis avec le signal reçu}\]
 %\[ R_xx(t) \text{L'autocorrélation du signal émis}\]
 %\[ b(t) \text{b est le bruit contenu dans la mesure}\]
\end{itemize}

Pour lancer le générateur de code, il faut initialiser en mettant le registre 1 à 1.La séquence de données aléatoires vont être servi pour créer un signal rectangulaire de période $ {T_s=(2^{n}-1)T_H} $ ($ f_H$ étant la  fréquence d'horloge utilisé par l'ensemble des composant $T_H=\frac{1}{f_H}$), sur une période $T_s$ on compte $2^{n-1}$ digit $2^{n-1}-1$ digit 0. 
Le temps d'acquisition doit être impérativement le multiple de la période de répétition de la séquence pseudo-aléatoire.
%F.de COULON Théorie et traitement des signaux
%% a faire le principe de pseudo bruit


\section{Etat de l'art de la version actuelle du sondeur}

\fig{0.8}{Emetteurv2.pdf}{Version actuelle de l'émetteur}
\fig{0.70}{Recepteurv2.png}{Version actuelle de du récepteur}
\newpage
\section{Étude matérielle du sondeur}

\subsection{Cahier de charge du sondeur}
Ce nouvel équipement doit permettre la mesure du canal de propagation dans la gamme
de fréquences 1-3 GHz, gamme dans laquelle opérera la troisième génération de systèmes radiomobiles.

Cet appareil doit permettre la mesure des effets du canal de propagation a l'intérieur des bâtiments et doit pouvoir fonctionner avec une alimentation électrique autonome. De ce fait, il devra être éminemment plus compact et moins gourmand que le sondeur du projet MARATHON, dont les différents éléments remplissaient une camionnette. Une diminution d’un facteur dix de ces deux paramètres est souhaitable.

Ensuite, bien que cet équipement soit en premier lieu dédié aux mesures dans les
bâtiments, il a été conçu pour pouvoir désadapter aux mesures en contextes macro et micro- cellulaire. De ce fait il doit permettre l’acquisition de réponses impulsionnelles ayant une très bonne résolution temporelle, afin de distinguer des échos très rapprochés comme c'est le cas dans les bâtiments. Il doit également permettre l’acquisition de réponses impulsionnelles assez longues dans des environnements plus vastes.

Cet équipement doit permettre l’analyse des décalages en fréquences Doppler, ainsi que de certains phénomènes pour lesquels la connaissance de la phase est primordiale.
L’identification que l’instrument réalise doit donc être celle des réponses impulsionnelles complexes du canal de propagation. 

\subsection{Émetteur du sondeur}

L'émetteur du sondeur produit un signal modulé par changement de phase binaire de fréquence centrale 2GHz.
\fig{0.36}{emetteur.png}{Vue synoptique de l'émetteur}\\

\begin{enumerate}
\item Générateur de séquence aléatoire est fait dans un module dans le PLA en utilisant l'algorithme détaillé dans cette section.\pageref{gene code pseudo}
\item Mélangeur de signaux HF coupler avec le O.L et générateur de séquence pour réaliser une modulation par changement de phase binaire(BPSK). La puissance du signal de sortie est de l'ordre de grandeur de 1W.
\item Amplificateur Radio Fréquence amplifie la puissance jusqu'à 30W.
\item Antennes d'émissions
\item Horloge d'atome Rubidium délivre une fréquence de 5MHz,sa précision plus important que l'horloge à quartz.
\item Oscillateur O.L.1 synthèse l'horloge du générateur de séquence (50 MHz,25 MHz et 12.5 MHz) et O.L.2 pour générer la fréquence de la porteuse qui peut être réglé de 900MHz à 2,5GHz.

La carte d'acquisition peut fonctionner jusqu'à la fréquence de 100MHz. Afin de respecter le théorème de shannon la fréquence utilisée est de 50 MHz pour générer la séquence aléatoire avec un débit de 50Mbit/s. 
\end{enumerate}
La consommation global estimé de l'émetteur est de l'ordre de grandeur de 500W.
%http://www.oscilent.com/esupport/TechSupport/ReviewPapers/IntroQuartz/vigcomp.htm
\subsection{Récepteur du sondeur}

Le récepteur est la partie de la chaine d'instrumentation la plus complexe. Actuellement il est constitué de deux partie, une partie réalise la fonction de conversion sur le signal reçu et une deuxième partie fait l'acquisition numérique du signal.

\fig{0.47}{recepteur.png}{Vue synoptique du récepteur}
\subsubsection{Partie conversion analogique-numérique}

La partie conversion permet de filtrer le signal reçu et amplifie la puissance du signal et ramène la fréquence centrale à 280MHz pour démoduler et faire une conversion analogique-numérique.
Cette partie est composée des éléments suivants:
\begin{enumerate}
\item Filtre passe-bande qui permet de récupérer les fréquences au tour de la fréquence porteuse et éliminer le reste
\item Amplificateur R.F permet d'amplifier la puissance du signal, il est conçu de façon à obtenir le meilleur rapport signal/bruit possible.
\item Atténuateur relié avec CAG pour protéger les éléments sensibles du de la partie acquisition.
\item CAG contrôle automatiquement le niveau d'amplitude des signaux reçus , pour que la conversion analogique-numérique se fasse sans dégradation. Il faut que le niveau de ces signaux doit être quasiment constant. Un interrupteur est inséré dans l'amplificateur, lorsque le niveau est trop important il coupe la partie chaine d'acquisition du circuit. 
\item Mélangeur est la pièce maîtresse du récepteur superhétérodyne. On applique à ses entrées les signaux de fréquences $F_{o}$ provenant de l'antenne, et $F_{OL}$ provenant de l'oscillateur local 8. Ce dernier produit un signal, modulé en amplitude ou fréquence, de fréquence $ F_{LO}$ proche de $ F_{o}$. On retrouve en sortie du mélangeur des signaux non seulement à $ F_{o}$ et $ F_{OL}$ mais aussi à $ F_{o}+F_{LO}$ et $ |F_{o}-F_{LO}|$. Ceci a pour but de décaler la fréquence tout en gardant l'allure du spectre.
\item Filtre FI permet de conserver uniquement la fréquence $ |F_{o}-F_{LO}|$. On obtiens une nouvelle fréquence centrale appelé la fréquence intermédiaire. Il y a également un filtre et un démodulateur pour démoduler le signal reçu.
\item La commande permet de contrôler un commutateur reliant soit l'entrée sur 1 ou 2 antennes soit l'entrée à une source de puissance étalon pour étalonner la chaîne de mesure. 
\item Oscillateur à quartz asservi a des caractéristiques proche de l'oscillateur à rubidium. De plus il a particularité de pouvoir être décalé en fréquence de l'ordre $1.10^{-11}$. Cette fonctionnalité permet de synchroniser la fréquence d'oscillateur de l'émetteur avec celle du récepteur avec une erreur relative de $1.10^{-11}$. 
\item Oscillateur local génère une fréquence qui est utilisé par le mélangeur pour transposer la fréquence centrale à 280 MHz.


\end{enumerate}


\subsubsection{Partie acquisition numérique}

Un mini Ordinateur permet de régler les paramètre du récepteur, tel que la sélections des antennes, réglage de l'oscillateur,rythme d'échantillonnage de I et Q,longueur de la séquence de code pseudo-aléatoire.
Deux carte d'acquisition sont utilisé pour réaliser la numérisation rapide pour les voies en phase et en quadrature. Et des cartes d'acquisitions d'information de C.A.G 
Le signal échantillonné par les antennes est enregistré dans un disque dure externe ainsi que les données sur imperfection du démodulateur et sera traité après les mesures par une station de travaille (SUN ou VAX 3400). Aucun calcul n'est fait dans cette partie, tous les traitement sont fait dans la partie de la chaine de traitement.
Ces donnés sont enregistré dans un fichier composé de 5 bloc contenants les informations suivantes:
\begin{itemize}[label=\ding{93}, font=\large \color{lightgray}]
\item Paramètre de la mesure
\item Les mesures code acquis au cours de l'échantillonnage
\item Distance ou localisation
\item Paramètre C.A.G
\item Échantillons "bruts" des acquisition des voies en phase et en quadrature.
\end{itemize}
Le micro-ordinateur calcule après acquisition de donnés la réponse impulsionnelle ou le spectre de la séquence mesurée.
Un mode de test en temps réel permet de visualiser les courbes en temps réel et affiche les diverses informations utiles pour l'état actuelle du récepteur.
\newpage
\subsubsection{Carte d'acquisition}

Les entrées de la carte est protection contre une surtension. Le front-end fournie une protection qui vas coupler automatiquement l'impédance de 50 $\Omega$ à 1 M$\Omega$ si le signal dépasse de ±5 V DC.
\fig{0.7}{protectionca.png}{La tension maximale protégée par la carte}

Les caractéristiques de la carte d'acquisition.
\fig{0.6}{temps_de_monte.png}{Le temps de monté et la bande passante de la carte}
\fig{0.43}{vitesse_acqui.png}{La vitesse d'acquisition de la carte}

\paragraph{Trigger Externe}

La carte d'acquisition permet aux utilisateur de choisir la pleine échelle(Full scale en anglais) du trigger external à partie de ces ensembles [0.5, 1, 2, 5V]. Ensuite le niveau trigger externe peut être mis entre $\pm$0.5* FS.
Lorsque la carte est en cours d'acquisition de donnée, le port TR OUT fournie une impulsion. Quand on arrête l'acquisition l'impulsion s'arrête.  

\paragraph{Horloge externe}
%Clock in
Deux modes de fonctionnent peut être choisis: Mode fréquence d'entrée continue qui fonctionne sous une fréquence entre 20 MHz and 2 GHz
Mode start\textbackslash stop fonctionne avec une fréquence d'entrée moins de 1GHz. Sous une tension d'au moins de 1V d'amplitude crête à crête.
Avant de lancer l'opération d'acquisition de donnée, il est intéressant de réaliser d'abord une auto-calibration
La transmission des données est réalisé par la communication DMA avec un débit de 100 MB/s.
%The continuous external clock mode (clockType = 1)
\newpage
\end{document}
