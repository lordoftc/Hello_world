\chapter{Conclusion}

J'ai tenté, tout au long de ce rapport, de présenter la méthode de caractérisation de matériaux et de manière globale tout ce qui a été réalisé durant ce stage de 2 mois au sein du laboratoire SATIE. Ce stage fut très enrichissant du point pédagogique car en plus d'avoir pu mettre en application un certain formalisme vu dans l'UE 434 relatif aux hyperfréquences,  avoir découvert le logiciel de simulation par éléments finis HFSS qui nous sera très utile pour la suite du cursus universitaire compte tenu du choix du master 2 recherche.\\
J'ai pu également redécouvrir la vie qui règne dans un laboratoire du point de vue du travail et de la coopération entre les différents protagonistes d'un laboratoire(chercheurs, doctorants, personnel administratif,...) mais aussi l'ambiance durant les heures de pauses.\\
Malheureusement je nourris un certain regret de n'avoir pu obtenir les données de la bentonite et de n'avoir pas réussi à résoudre le problème inverse.\\